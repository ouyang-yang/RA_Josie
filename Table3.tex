\begin{table} \scalebox{0.85}{ \begin{tabular}{l*{6}{c}} \hline \hline
                &\multicolumn{1}{c}{(1)}&\multicolumn{1}{c}{(2)}&\multicolumn{1}{c}{(3)}&\multicolumn{1}{c}{(4)}&\multicolumn{1}{c}{(5)}&\multicolumn{1}{c}{(6)}\\
\hline \\ \textbf{\underline {Panel A: Individual Characteristics}}&&&&&&& \\ & \multicolumn{6}{c}{Dependent Variable: Majority Method} \\ \cmidrule(lr){2-7} \\
Age             &   -0.008&         &         &         &         &         \\
                &  (0.005)&         &         &         &         &         \\
[1em]
Gender          &         &    0.035&         &         &         &         \\
                &         &  (0.093)&         &         &         &         \\
[1em]
Married         &         &         &   -0.149&         &         &         \\
                &         &         &  (0.087)&         &         &         \\
[1em]
 High SES       &         &         &         &    0.099&         &         \\
                &         &         &         &  (0.109)&         &         \\
[1em]
High Educational Attainment&         &         &         &         &    0.262&         \\
                &         &         &         &         &  (0.273)&         \\
[1em]
Current University Student&         &         &         &         &         &    0.174\\
                &         &         &         &         &         &  (0.092)\\
\hline
R-squared       &    0.005&    0.000&    0.005&    0.002&    0.002&    0.005\\
\hline \\ \textbf{\underline {Panel B: Economic Experiences}}&&&&&&& \\ & \multicolumn{6}{c}{Dependent Variable: Majority Method} \\ \cmidrule(lr){2-7} \\
UrbanBackground &    0.091&         &         &         &         &         \\
                &  (0.075)&         &         &         &         &         \\
[1em]
Migrated Internally Within China in Childhood&         &   -0.186&         &         &         &         \\
                &         &  (0.067)&         &         &         &         \\
[1em]
Province GDP Growth(Average 92-19)&         &         &   -0.216&       \documentclass[]{article}
\usepackage{amssymb}
\usepackage{adjustbox}
\usepackage{booktabs}
\usepackage{changepage}

\begin{document}

\begin{table}
\begin{adjustbox}{width=\columnwidth,center}
\begin{tabular}{l*{6}{c}} \hline \hline
                &\multicolumn{1}{c}{(1)}&\multicolumn{1}{c}{(2)}&\multicolumn{1}{c}{(3)}&\multicolumn{1}{c}{(4)}&\multicolumn{1}{c}{(5)}&\multicolumn{1}{c}{(6)}\\
\hline \\ \textbf{\underline {Panel A: Individual Characteristics}}&&&&&&& \\ & \multicolumn{6}{c}{Dependent Variable: Majority Method} \\ \cmidrule(lr){2-7} \\
Age             &   -0.008&         &         &         &         &         \\
                &  (0.005)&         &         &         &         &         \\
[1em]
Gender          &         &    0.035&         &         &         &         \\
                &         &  (0.093)&         &         &         &         \\
[1em]
Married         &         &         &   -0.149&         &         &         \\
                &         &         &  (0.087)&         &         &         \\
[1em]
 High SES       &         &         &         &    0.099&         &         \\
                &         &         &         &  (0.109)&         &         \\
[1em]
High Educational Attainment&         &         &         &         &    0.262&         \\
                &         &         &         &         &  (0.273)&         \\
[1em]
Current University Student&         &         &         &         &         &    0.174\\
                &         &         &         &         &         &  (0.092)\\
R-squared       &    0.005&    0.000&    0.005&    0.002&    0.002&    0.005\\
\hline \\ \textbf{\underline {Panel B: Economic Experiences}}&&&&&&& \\ & \multicolumn{6}{c}{Dependent Variable: Majority Method} \\ \cmidrule(lr){2-7} \\
Urban Background &    0.091&         &         &         &         &         \\
                &  (0.075)&         &         &         &         &         \\
[1em]
Migrated Internally Within China in Childhood       &         &   -0.186&         &         &         &         \\
                &         &  (0.067)&         &         &         &         \\
[1em]
Province GDP Growth(Average 92-19)&         &         &   -0.216&         &         &         \\
                &         &         &  (3.526)&         &         &         \\
[1em]
Province Imports/GDP(Average 92-19)&         &         &         &   -0.288&         &         \\
                &         &         &         &  (0.278)&         &         \\
[1em]
Province Exports/GDP(Average 92-19)&         &         &         &         &   -0.625&         \\
                &         &         &         &         &  (0.230)&         \\
[1em]
Province FDI/GDP(Average 92-19)&         &         &         &         &         &   -0.100\\
                &         &         &         &         &         &  (0.083)\\
R-squared       &    0.002&    0.007&    0.000&    0.003&    0.012&    0.003\\
\hline
Observations    &  489&  489&  489&  489&  489&  489\\
        \hline\hline        
\end{tabular}  
\end{adjustbox}
\\
\begin{adjustwidth}{-1cm}{-1cm}             
    \begin{scriptsize}       
    Each panel-column reports results from a single OLS regression of Majority Method on the variable described in the row heading. Individual-level data comes from information shared by survey respondents in mainland China who were recruited with the assistance of the online platform Wenjuanxing (see Section I.B and online Appendix B.1 for further details about the data collection), whereas province-level data comes from China's National Bureau of Statistics and the UNDP National Human Development Report. Majority Method is the preference over a method to assign a decision for a group in the incentivized task, with 1 corresponding to a strong preference for a decision by an authority designated by the experimenter, and 5 corresponding to a strong preference for majority rule. Province-level information is linked to a respondent based on her reported province of origin. Online Appendix A.1 provides further details about the definition and the scale of each variable in the row heading. Panel A presents robust standard errors in parentheses, while Panel B presents robust standard errors clustered at the province of origin of the respondent.           
        \end{scriptsize} 
    \end{adjustwidth} 
\end{table}
\end{document}  &         &         \\
                &         &         &  (3.526)&         &         &         \\
[1em]
Province Imports/GDP(Average 92-19)&         &         &         &   -0.288&         &         \\
                &         &         &         &  (0.278)&         &         \\
[1em]
Province Exports/GDP(Average 92-19)&         &         &         &         &   -0.625&         \\
                &         &         &         &         &  (0.230)&         \\
[1em]
Province FDI/GDP(Average 92-19)&         &         &         &         &         &   -0.100\\
                &         &         &         &         &         &  (0.083)\\
\hline
R-squared       &    0.002&    0.007&    0.000&    0.003&    0.012&    0.003\\
Observations    &  489.000&  489.000&  489.000&  489.000&  489.000&  489.000\\
        \hline\hline        \end{tabular}}            \begin{adjustwidth}{-1cm}{-1cm}             \begin{scriptsize}       Each panel-column reports results from a single OLS regression of Majority Method on the variable described in the row heading. Individual-level data comes from information shared by survey respondents in mainland China who were recruited with the assistance of the online platform Wenjuanxing (see Section I.B and online Appendix B.1 for further details about the data collection), whereas province-level data comes from China's National Bureau of Statistics and the UNDP National Human Development Report. Majority Method is the preference over a method to assign a decision for a group in the incentivized task, with 1 corresponding to a strong preference for a decision by an authority designated by the experimenter, and 5 corresponding to a strong preference for majority rule. Province-level information is linked to a respondent based on her reported province of origin. Online Appendix A.1 provides further details about the definition and the scale of each variable in the row heading. Panel A presents robust standard errors in parentheses, while Panel B presents robust standard errors clustered at the province of origin of the respondent.           \end{scriptsize} \end{adjustwidth} \end{table}
\documentclass[]{article}
\usepackage{amssymb}
\usepackage{adjustbox}
\usepackage{booktabs}
\usepackage{changepage}

\begin{document}

\begin{table}
\begin{adjustbox}{width=\columnwidth,center}
\begin{tabular}{l*{6}{c}} \hline \hline
                &\multicolumn{1}{c}{(1)}&\multicolumn{1}{c}{(2)}&\multicolumn{1}{c}{(3)}&\multicolumn{1}{c}{(4)}&\multicolumn{1}{c}{(5)}&\multicolumn{1}{c}{(6)}\\
\hline \\ 
\textbf{\underline{Panel A: Individual Characteristics}} &&&&&& \\ 
& \multicolumn{6}{c}{Dependent Variable: Majority Method} \\ \cmidrule(lr){2-7} 
Age             &   -0.008&         &         &         &         &         \\
                &  (0.005)&         &         &         &         &         \\
[1em]
Gender          &         &    0.035&         &         &         &         \\
                &         &  (0.093)&         &         &         &         \\
[1em]
Married         &         &         &   -0.149&         &         &         \\
                &         &         &  (0.087)&         &         &         \\
[1em]
 High SES       &         &         &         &    0.099&         &         \\
                &         &         &         &  (0.109)&         &         \\
[1em]
High Educational Attainment&         &         &         &         &    0.262&         \\
                &         &         &         &         &  (0.273)&         \\
[1em]
Current University Student&         &         &         &         &         &    0.174\\
                &         &         &         &         &         &  (0.092)\\
R-squared       &    0.005&    0.000&    0.005&    0.002&    0.002&    0.005\\
\hline \\ \textbf{\underline {Panel B: Economic Experiences}}&&&&&&& \\ & \multicolumn{6}{c}{Dependent Variable: Majority Method} \\ \cmidrule(lr){2-7} \\
Urban Background &    0.091&         &         &         &         &         \\
                &  (0.075)&         &         &         &         &         \\
[1em]
Migrated Internally Within China in Childhood       &         &   -0.186&         &         &         &         \\
                &         &  (0.067)&         &         &         &         \\
[1em]
Province GDP Growth(Average 92-19)&         &         &   -0.216&         &         &         \\
                &         &         &  (3.526)&         &         &         \\
[1em]
Province Imports/GDP(Average 92-19)&         &         &         &   -0.288&         &         \\
                &         &         &         &  (0.278)&         &         \\
[1em]
Province Exports/GDP(Average 92-19)&         &         &         &         &   -0.625&         \\
                &         &         &         &         &  (0.230)&         \\
[1em]
Province FDI/GDP(Average 92-19)&         &         &         &         &         &   -0.100\\
                &         &         &         &         &         &  (0.083)\\
R-squared       &    0.002&    0.007&    0.000&    0.003&    0.012&    0.003\\
\hline
Observations    &  489&  489&  489&  489&  489&  489\\
        \hline\hline        
\end{tabular}  
\end{adjustbox}
\\
\begin{adjustwidth}{-1cm}{-1cm}             
    \begin{scriptsize}       
    Each panel-column reports results from a single OLS regression of Majority Method on the variable described in the row heading. Individual-level data comes from information shared by survey respondents in mainland China who were recruited with the assistance of the online platform Wenjuanxing (see Section I.B and online Appendix B.1 for further details about the data collection), whereas province-level data comes from China's National Bureau of Statistics and the UNDP National Human Development Report. Majority Method is the preference over a method to assign a decision for a group in the incentivized task, with 1 corresponding to a strong preference for a decision by an authority designated by the experimenter, and 5 corresponding to a strong preference for majority rule. Province-level information is linked to a respondent based on her reported province of origin. Online Appendix A.1 provides further details about the definition and the scale of each variable in the row heading. Panel A presents robust standard errors in parentheses, while Panel B presents robust standard errors clustered at the province of origin of the respondent.           
        \end{scriptsize} 
    \end{adjustwidth} 
\end{table}
\end{document}